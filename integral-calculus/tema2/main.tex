\documentclass[14pt, a4paper, oneside]{extarticle}
\usepackage{amsmath, amssymb, amsthm}
\setlength{\parindent}{0pt}

\begin{document}
\subsection*{Resuelva y realice lo que se indica}

\begin{enumerate}
	\item Área entre curvas
	      Encuentra el área encerrada entre las curvas:
	      \begingroup
	      \Large
	      \begin{equation*}
		      y=x^3   y=x
	      \end{equation*}
	      \endgroup

	      Instrucciones:
	      \begin{itemize}
		      \item Determina los puntos de intersección.
		      \item Grafica las funciones para verificar qué curva está arriba.
		      \item Plantea la integral definida que represente el área encerrada.
		      \item Calcula el valor exacto del área.
	      \end{itemize}

	\item Sólido de revolución
	      Se gira alrededor del eje x la región encerrada entre las curvas:
	      \begingroup
	      \Large
	      \begin{equation*}
		      y = \sqrt{x} \text{ y } y = 0 \text{ desde } x = 0 \text{ hasta } x = 4
	      \end{equation*}
	      \endgroup

	      Encuentra el volumen del sólido generado mediante el método de discos.

	      Instrucciones:
	      \begin{itemize}
		      \item Dibuja el perfil de la región a girar.
		      \item Usa la fórmula del volumen con discos.
		      \item Calcula la integral y simplifica el resultado.
	      \end{itemize}

	\item Integral impropia
	      Evalúa la siguiente integral impropia:

	      \begingroup
	      \Large
	      \begin{equation*}
		      \int_{1}^{\infty} \frac{1}{x^2} \ dx
	      \end{equation*}
	      \endgroup

	      Instrucciones:
	      \begin{itemize}
		      \item Identifica por qué es una integral impropia.
		      \item Evalúa el límite correspondiente.
		      \item Determina si la integral converge o diverge, y encuentra su valor si converge.

	      \end{itemize}


\end{enumerate}
\end{document}
