\documentclass[12pt]{article}
\usepackage[a4paper, margin=2.4cm]{geometry}
\usepackage{amsmath}
\usepackage{amssymb}
\usepackage{graphicx}
\usepackage[labelformat=empty]{caption}

\graphicspath{ {./assets/} }

\begin{document}

\mathversion{bold}
\section{Definicion de Espacio Vectorial}
Un espacio vectorial es un conjunto no vacío V de objetos, llamados vectores,
en el que están definidas dos operaciones, llamadas suma y multiplicación por
escalares (números reales). (David C. Lay, 2007, p. 217)

Estas operaciones deben cumplir con un conjunto específico de propiedades o axiomas, como la asociatividad, conmutatividad, distributividad y existencia de un elemento neutro. La suma de dos vectores debe resultar en otro vector dentro del mismo espacio, y el producto de un escalar por un vector también debe dar como resultado un vector del espacio.

\subsection{Notacion}
Si $ x $ y $ y $ están en $ V $ y si a es un número real, entonces la suma se escribe como
$ x + y $ y el producto escalar de $ a $ y $ x $ como $ ax $. (Stanley I. Grossman, 2007, p. 282) \medbreak

Ejemplo:
Sea $ x, y \in V $ vectores pertenecientes a un espacio vectorial, y $ a \in \mathbb{R} $ un escalar perteneciente a los numeros reales.


\begin{itemize}
  \item Suma: 
    {
     \Large
    \begin{align*}
      x + y = (x_1, + y_1, x_2 + y_2, \dotsb , x_n + y_n)
    \end{align*}
    }
  \item Multiplicación por un escalar:
    {
      \large
      \begin{align*}
        ax = (ax_1, ax_2, \dotsb , ax_3)
      \end{align*}
    }
\end{itemize}

\subsection{Axiomas de un espacio vectorial}
\begin{enumerate}
  \item Si $ x \in V  $ y $ y \in V $, entonces $ x + y \in V $ \textbf{(cerradura bajo la suma)}.
  \item Para todos $ x , y , z $ en $ V $, $ (x + y) + z = x + (y + z) $ \textbf{(ley asociativa de la suma de vectores)}.
  \item Existe un vector $ 0 \in V $  tal que para todo $ x \in V, x + 0  = 0 + x = x $  (el 0 se llama \textbf{vector cero} o \textbf{identico aditivo}).
  \item Si $ x \in V $, existe un vector $ -x $ en $ \in V $ tal que $ x + (-x) $ = 0 ($ -x $ se llama \textbf{inverso aditivo de x}).
  \item Si $ x $ y $ y $ estan en $ V $, entonces $ x + y = y + x $ \textbf{(ley conmutativa de la suma de vectores)}.
  \item Si $ x \in V $ y $ \alpha $ es un escalar, entonces $ \alpha x \in V $ \textbf{(cerradura bajo la multiplicación por un escalar)}
  \item Si $ x $ y $ y $ estan en $ V $ y $ \alpha $ es un escalar, entonces $ \alpha (x + y) = \alpha x + \alpha y $ \textbf{(primera ley distributiva)}
  \item Si $ x \in V $ y $ \alpha $ y $ \beta $ son escalares, entonces $ (\alpha + \beta)x = \alpha x + \beta x $ \textbf{(segunda ley distributiva)}.
  \item Si $ x \in V $ y $ \alpha $ y $ \beta $ son escalares, entonces $ \alpha (\beta x) = (\alpha \beta )x $ \textbf{(ley asociativa de la multiplicación por escalares)}.
  \item Para cada vector $ x \in V , 1x = x$
\end{enumerate}
(Stanley I. Grossman, 2007, 282).

\subsection{Ejemplo}
El conjunto
{\large \[ \mathbb{R}^{3} = \{ (x, y, z) | z, y, z \in \mathbb{R} \} \]}
con las operaciones usuales:
\begin{itemize}
  \item Suma
    {\large \[ (x_1, y_1, z_1 ) + (x_2, y_2, z_2) = (x_1 + x_2, y_1 + y_2, z_1 + z_2) \]}
  \item Producto por escalar:
    {\large \[ a(x, y, z) = (ax, ay, az) \]}
\end{itemize}

y cumple con los 10 axiomas de espacio vectorial, entonces $ \mathbb{R}^3 $ \textbf{es un espacio vectorial}

Ejemplo grafico:
\begin{figure}[htbp]
  \begin{center}
    \includegraphics[width=0.50\textwidth]{octant}
  \end{center}
\end{figure}

\newpage

\section{Definicion de Subespacio Vectorial}
Un \textbf{subespacio} de un espacio vectorial V es un subconjunto H de V que tiene tres propiedades:
\begin{enumerate}
  \item El vector cero de $ V $  está en $ H^2 $
  \item $ H $ es cerrado bajo la suma de vectores. Esto es, para cada $ u $ y $ v $ en $ H $, la suma de $ u + v $ está en $ H $
  \item $ H $ es cerrado bajo la multiplicación por escalares. Esto es, para cada $ u $ en $ H $ y cada escalar $ a $, el vector $ au $ está en $ H $
\end{enumerate}
(David C. Lay, 2007, p. 220).
\medbreak
Así, todo subespacio es un espacio vectorial. De manera recíproca, todo espacio
vectorial es un subespacio (de sí mismo o posiblemente de espacios mayores). El término subespacio es usado cuando se 
consideran por lo menos dos espacios, con uno dentro
de otro, y la frase subespacio de $ V $ identifica a $ V $ como el espacio más grande. (David C. Lay, 2007, p. 220). Se puede decir que el subespacio $ H $ hereda las operaciones del espacio vectorial “padre” $ V $. (Stanley I. Grossman, 2007, p. 293)

\begin{figure}[htbp]
  \begin{center}
    \includegraphics[width=0.95\textwidth]{subspace}
    \smallbreak
  \end{center}
  \caption{Un subespacio de $ V $}
\end{figure}

\subsection{Reglas de Cerradura}
Un subconjunto no vacío $ H $ de un espacio vectorial $ V $ es un subespacio de $ V $ si se cumplen las dos reglas de cerradura:

\begin{enumerate}
  \item Si $ u \in H $ y $ v \in H $, entonces $ u + v \in H $.
  \item Si $ u \in H $, entonces $ au \in H $ para todo escalar $ a $
\end{enumerate}

Es obvio que si $ H $ es un espacio vectorial, entonces las dos reglas de cerradura deben cumplirse. (Stanley I. Grossman, 2007, p. 293)

\subsection{Ejemplo}
Considera el conjunto:
{\large \[ W = \{ (x, y) \in \mathbb{R}^2 | y = 0 \} \]}
Este conjunto contiene todos los vectores del eje $ x $, es decir: 
{\large \[ W = \{ (x, 0) | x \in \mathbb{R} \} \]}
y si es \textbf{un subespacio vectorial} de $ \mathbb{R}^2 $

\subsubsection{Verificacion}
Tomemos dos vectores de $ W $:
{\large \[ u = (3, 0), \quad v = (-2, 0) \]}

\begin{enumerate}
  \item El vector cero pertenece al conjunto
    {\large \[ (0, 0) \in W \quad porque \quad  y = 0 \]}
  \item Cerradura bajo la suma
    {\large \[ u + v = (3 + (-2), 0 + 0) = (1, 0) \]}
    Como su segunda componente es cero, está en $ W $
  \item Cerradura bajo el producto por escalar: 
    sea $ a = 4 $
    {\large \[ au = 3(3, 0) = (12, 0) \]}
    También cumple $ y $ = 0, por lo tanto $au \in W$.
\end{enumerate}
 
\newpage

\begin{figure}[htbp]
  \begin{center}
    \includegraphics[width=0.80\textwidth]{r2}
  \end{center}
  \caption{El subespacio $ W = \{ (x, y) \in \mathbb{R}^2 | y = 0 \} $ }
\end{figure}

\section{Referencias}
\begin{itemize}
  \item David C. Lay. (2007). Álgebra lineal y sus aplicaciones. Mexico. Pearson Educación
  \item Stanley I. Grossman. (2007) .Álgebra Lineal. Mexico. The McGraw-Hill
\end{itemize}


\end{document}
