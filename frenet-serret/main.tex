\documentclass[12pt] {article}
\usepackage[a4paper]{geometry}
\usepackage{amsmath}
\usepackage[utf8]{inputenc}
\usepackage{svg}
\usepackage{setspace}
\usepackage{bm}
\usepackage{amssymb}
\usepackage{graphicx}
\usepackage{helvet}

\DeclareSymbolFont{letters}{OML}{ztmcm}{m}{it}
\DeclareSymbolFontAlphabet{\mathnormal}{letters}

\graphicspath{ {./assets/} }
\renewcommand{\familydefault}{\sfdefault}
\onehalfspacing
 
\begin{document}
\section*{Formulas de Frenet-Serret}
\subsection*{Jean Frenet}

Jean Frédéric Frenet (7 de febrero de 1816 - 12 de junio de 1900) fue un
famoso matemático que introdujo la Teoría de Curvas junto a Joseph Serret. 
En reconocimiento a su trabajo, se denomina a la base espacial definida
por los vectores tangente, normal y binormal, triedro de Frenet-Serret.\medbreak

Nació en Périgueux en 1816 y en el año 1840 ingresó en L’École Normale Superieure, 
más tarde continuó sus estudios en Toulouse, ciudad en la que redactó su tesis doctoral
durante 1847. Un fragmento de la mencionada tesis alberga la teoría de curvas en el espacio, 
incluyendo las fórmulas que actualmente son conocidas como ‘fórmulas de Frenet – Serret’.
Frenet aportó seis de dichas fórmulas, mientras que Serret proporcionó las nueve restantes.
Cabe señalar que Frenet publicó este apartado de su tesis en el ‘Journal de mathématique pures et appliques’, en el año 1852. \medbreak

Frenet llegó a ser profesor en Toulouse y, después, en 1848, ocupó un puesto de docente de matemáticas en Lyon.
Además, también fue director del observatorio astronómico, donde, como tal, dirigió las observaciones meteorológicas. 

\subsection*{Joseph Alfred Serret}
Joseph Alfred Serret (París, Francia, 30 de agosto de 1819 - Versalles, Francia, 2 de marzo de 1885),
más conocido como Joseph Serret, fue un matemático francés, conocido por desarrollar junto a Jean Frenet 
la teoría de curvas. Editó los trabajos de Lagrange —publicados en catorce volúmenes entre 1867 y 1892
y realizó la quinta edición de los de Monge en 1850. Una de sus principales obras fue el manual Cours
d'Algèbre supérieure, editado en dos tomos. \medbreak

Joseph-Alfred Serret era hijo de Pierre Antoine Serret y de Marie Virginie Tessier. Nacido en la calle de Sant Honoré de París, 
realizó estudios superiores en la Escuela Politécnica de 1838 a 1840,que continuó en la Escuela de la Tabacalera. Renunció a su 
trabajo como ingeniero de la tabacalera para regresar a París, donde se convirtió en examinador en el Colegio de Santa Bárbara. 
En 1847 obtuvo el doctorado en ciencias matemáticas en la Facultad de Ciencias de París, y un año después se convirtió en examinador 
de admisión en la Escuela Politécnica, cargo que ocupó hasta 1862. 

\subsection*{Introduccion a las formulas de Frenet-Serret}

En geometría diferencial, las fórmulas de Frenet-Serret describen las propiedades cinemáticas de una partícula 
que se mueve a lo largo de una curva diferenciable en el espacio euclídeo tridimensional $ \mathbb{R}^{3} $, o las propiedades 
geométricas de la propia curva módulo desplazamientos de la misma. Más específicamente, las fórmulas describen las derivadas de cada vector unitario 
del triedro de Frenet-Serret en términos de los otros dos. Las fórmulas llevan el nombre de los dos matemáticos franceses que las descubrieron de forma 
independiente: Jean Frédéric Frenet, en su tesis de 1847, y Joseph Alfred Serret, en 1851. La notación vectorial y el álgebra lineal que se emplean 
actualmente para escribir estas fórmulas aún no estaban disponibles en el momento de su descubrimiento. \medbreak

En cada punto de una curva diferenciable, el triedro de Frenet-Serret es una base ortonormal de $ \mathbb{R}^{3} $ formada por los siguientes vectores: 

\begin{itemize}
  \item $ T $ es el vector unitario tangente a la curva, apuntando en la dirección del movimiento.
  \item $ N $ es el vector unitario normal, en el sentido de la derivada de $ T $ con respecto del parámetro longitud de arco.
  \item $ B $ es el vector unitario binormal, el producto vectorial de $ T $ y $ N $.
\end{itemize}

Las formulas de Frenet-Serret son:
{
  \Large
\begin{align*}
\frac{d\mathbf{T}}{ds} &= \kappa \mathbf{N} \\ \\
\frac{d\mathbf{N}}{ds} &= -\kappa \mathbf{T} + \tau \mathbf{B} \\ \\
\frac{d\mathbf{B}}{ds} &= -\tau \mathbf{N}
\end{align*}
}

Donde $ \frac{d}{ds} $ denota la derivada con respecto al parámetro longitud del arco, $ \kappa $ es la curvatura y $ \tau $ es la torsión de la curva, dos magnitudes escalares. 
Intuitivamente, la curvatura mide lo que dista una curva de ser una línea recta, mientras que la torsión mide lo que dista una curva de ser plana.

\begin{align*}
  \centering
  \includesvg[width=9cm, height=9cm]{./assets/Frenet.svg}
\end{align*}

Una curva alabeada; los vectores $ T $, $ N $ y $ B $; y el plano osculador atravesado por T y N

\subsection*{Enunciado}
Sea $ R(s) $ una curva en el espacio euclídeo, que representa el vector de posición de la partícula en función del tiempo. Las fórmulas 
de Frenet-Serret se aplican a curvas que no son degeneradas, es decir, que se curvan en todo punto. Más formalmente, 
se requiere que el vector de velocidad $ R'(s) $ y el vector de aceleración $ R''(s) $ no sean proporcionales. \medbreak

Además, se puede suponer que la curva está parametrizada por el arco (es decir, que $\lvert\lvert R(s) \rvert\rvert = 1 $ para todo $ s $ ), ya que cualquier parametrización de la curva da la misma curvatura en cada punto. \medbreak

En estas condiciones, el triedro de Frenet-Serret (o TNB) se define como: 
\begin{itemize}
  \item El vector unitario tangente $ T $ en un parametrización por el arco se define como: 

    {
    \Large
    \begin{align*}
      \mathbf{T} = \frac{d\mathbf{r}}{ds}
    \end{align*}
    }

  \item El vector unitario normal N a su vez se define como:

    {
    \Large
    \begin{align*}
      \mathbf{N} = \frac{
        \frac{d\mathbf{T}}{ds}
      }{
        \lvert\lvert \frac{d\mathbf{T}}{ds} \rvert\rvert
      }
    \end{align*}
    }
    de donde se sigue, ya que la longitud de $ \mathbf{T} $ es constantemente 1, que $ \mathbf{N} $ es perpendicular a $ \mathbf{T} $. Como
    $ \kappa = \lvert\lvert \frac{d\mathbf{T}}{ds} \rvert\rvert $ , obtenemos la primera fórmula de Frenet-Serret. \\ \\ \\

  \item El vector unitario binormal $ \mathbf{B} $ se define como el producto vectorial de $ \mathbf{T} $ y $ \mathbf{N} $:
    {
    \Large
    \begin{align*}
       \mathbf{B} = \mathbf{T} \times \mathbf{N} 
    \end{align*}
    }

\end{itemize}

\begin{align*}
  \centering
  \includesvg[width=15cm, height=15cm]{./assets/FrenetTN.svg}
\end{align*}

Los vectores $ \mathbf{T} $ y $ \mathbf{N} $ en dos puntos de una curva plana, una versión desplazada del segundo diedro (punteado) y
el cambio en $ \mathbf{T} $, $ \delta\mathbf{T} $. Si $ s $ es la distancia entre los puntos, en el límite d T d s  apuntará en la dirección N y
su magnitud —la curvatura— describe la velocidad de rotación del diedro. \medbreak

Las fórmulas de Frenet-Serret (o Teorema de Frenet-Serret) son: 
{
  \Large
\begin{align*}
  \frac{d\bm{T}}{ds} &= \qquad\qquad \kappa \bm{N} \\[1em]
  \frac{d\bm{N}}{ds} &= -\kappa \bm{T} \qquad\qquad\quad + \tau \bm{B} \\[1em]
\frac{d\bm{B}}{ds} &= \qquad\quad -\tau \bm{N}
\end{align*}
}

donde $ \kappa $  es la curvatura y $ \tau $ es la torsión. \medbreak

O, en notación matricial:
{
\Large
\begin{align*}
  \begin{bmatrix}
    \mathbf{T}' \\
    \mathbf{N}' \\
    \mathbf{B}' \\
  \end{bmatrix}
  =
  \begin{bmatrix}
    0 & \kappa & 0 \\
    -\kappa & 0 & \tau \\
    0 & -\tau & 0 \\
  \end{bmatrix}
  \begin{bmatrix}
    \mathbf{T} \\
    \mathbf{N} \\
    \mathbf{B} \\
  \end{bmatrix}
\end{align*}
}
Esta matriz es antisimétrica

\begin{align*} 
  \centering
  \includegraphics{Frenetframehelix}
\end{align*}
  El triedro de Frenet-Serret moviéndose a lo largo de una hélice . La $ \mathbf{T} $ está representada por la flecha azul,
la $ \mathbf{N} $ por la flecha roja, y la $ \mathbf{B} $ por la flecha negra.

\subsection*{Formulas en $ \mathbf{n} $ dimensiones}
Las fórmulas de Frenet-Serret fueron generalizadas a espacios euclídeos de dimensión superior por Camille Jordan en 1874. \medbreak

Supongamos que $ r(s) $ es una curva suave en $ \mathbb{R}^{n} $ , y que las primeras $ n $ derivadas de $ \mathbf{r} $
son linealmente independientes. Los vectores en el $ n $-edro de Frenet-Serret son una base ortonormal
construida aplicando el proceso de Gram-Schmidt a los vectores
($ \mathbf{r} $'(s), $ \mathbf{r} $''(s), ..., $ \mathbf{r}^{(n)} $(s)). 

El vector tangente unitario es el primer vector de Frenet $ e_1(s) $, y se define como: 

{
  \Large
  \begin{align*}
    e_1(s) = \frac{\overline{e_1}(s)}{\lvert\lvert \overline{e_1}(s) \rvert\rvert }
  \end{align*}
}
donde
{
  \Large
  \begin{align*}
    e_1(s) = \mathbf{r}'(s)
  \end{align*}
}


El vector normal, a veces llamado vector de curvatura, indica lo que dista la curva de ser una línea 
recta. Se define como: 
{
  \Large
  \begin{align*}
    \overline{e_2}(s) = \mathbf{r}^{n}(s) - (\mathbf{r}^{n}(s), e_1(s))e_1(s)
  \end{align*}
}

Su forma normalizada, el vector normal unitario, es el segundo vector de Frenet $ e_2(s) $ y se define como: 
{
  \Large
  \begin{align*}
    e_2(s) =  \frac{\overline{e_2}(s)}{\lvert\lvert \overline{e_2}(s) \rvert\rvert}
  \end{align*}
}

Los vectores tangente y normal en el punto s definen el plano osculador en el punto $ r(s) $. \medbreak

Los vectores restantes en el $ n $-edro (el binormal, trinormal, etc.) se definen de manera similar: 
{
  \Large
  \begin{align*}
    e_j(s) &=  \frac{\overline{e_j}(s)}{\lvert\lvert \overline{e_j}(s) \rvert\rvert} \\
    \overline{e_j}(s) &= \mathbf{r}^{(j)}(s) - \sum_{i=1}^{j-1}(\mathbf{r}^{(j)}(s), e_i(s))e_i(s)
  \end{align*}
}

El último vector del sistema se define como el producto vectorial de los primeros $ n-1 $  vectores: 
{
  \Large
  \begin{align*}
    e_n(s) = e_1(s) \times e_2(s) \times ... \times e_{n-2}(s) \times e_{n-1}(s)
  \end{align*}
}

Las siguientes funciones reales $ \chi_i(s) $ se denominan curvaturas generalizadas: 
{
  \Large
  \begin{align*}
    \chi_i(s) = \frac{\langle e'_i(s), e_{i+1}(s) \rangle}{\lvert\lvert \mathbf{r}'(s) \rvert\rvert}
  \end{align*}
}

Las fórmulas de Frenet-Serret, expresadas en lenguaje matricial, son:
{
  \Large
  \begin{align*}
   \begin{bmatrix}
     e_1'(s) \\
     \vdots \\ 
     e_n'(s) \\
   \end{bmatrix}  
   = \lvert\lvert \mathbf{r}'(s) \rvert\rvert \cdot
   \begin{bmatrix}
    0          & \chi_1(s) &                & 0 \\
    -\chi_1(s) & \ddots    & \ddots         &  \\
               & \ddots    & 0              &  \\
    0          &           & -\chi_{n-1}(s) & 0 \\
   \end{bmatrix}
   \begin{bmatrix}
      e_1(s) \\
      \vdots \\ 
      e_n(s) \\
   \end{bmatrix}
  \end{align*}
}

Nótese que la convención usada al definir las curvaturas y el sistema de Frenet en dimensión $ n $ no es universal. 
La curvatura superior $ \chi_{n-1} $ (también llamada torsión, en este contexto) y el último vector en el marco $ e_n $, 
difieren de la torsión habitual por el signo 

\subsection*{Ejemplo: Hélice circular}
Considera la curva:
{
  \Large
  \begin{align*}
    r(t) = (a\cos t, a \sin t, bt)
  \end{align*}
}
donde $ a $ y $ b $ son constantes; Calcule los vectores $ \mathbf{T}, \mathbf{N}, \mathbf{B} $, la curvatura $ \kappa $ 
y la torsión $ \tau $ y verifique las ecuaciones con las formulas de Frenet-Serret

\subsubsection*{Derivadas de $ r $}
{
  \large
  \begin{align*}
    r'(t) &= (-a \sin t, a \cos t, b) \\
    r''(t) &= (-a \cos t, -a \sin t, 0) \\
    r'''(t) &= (a \sin t, -a \cos t, 0) \\ 
    \lvert r'(t) \rvert &= \sqrt{ (-a \sin t)^2 + (a \cos t)^2 + b^2 } \\
    \lvert r'(t) \rvert &= \sqrt{a^2 + b^2}
  \end{align*}
}

\subsubsection*{Vector tangente unitario $ \mathbf{T} $}
{
  \large
  \begin{align*}
    \mathbf{T}(t) &= \frac{r'(t)}{\lvert r'(t) \rvert} \\
    \mathbf{T}(t) &= \frac{1}{\sqrt{a^2 + b^2}}(-a \sin t, a \cos t, b) 
  \end{align*}
}

su derivada con respecto a $ t $
{
  \large
  \begin{align*}
    \frac{d\mathbf{T}}{dt} &= \frac{1}{\sqrt{a^2 + b^2}}(-a \cos t, -a \sin t, 0) \\
  \end{align*}
}

su derivada con respecto a la longitud de arco $ s $ usamos $ \frac{d}{ds} = \frac{1}{\lvert r' \rvert} \frac{d}{dt} $
{
  \large
  \begin{align*}
    \frac{d\mathbf{T}}{ds} &= \frac{1}{\lvert r' \rvert}\frac{d\mathbf{T}}{dt} =
    \frac{1}{\sqrt{a^2 + b^2}} \cdot \frac{1}{\sqrt{a^2 + b^2}}(-a \cos t, -a \sin t, 0) \\
    \frac{d\mathbf{T}}{ds} &= \frac{1}{a^2 + b^2}(-a \cos t, -a \sin t, 0)
  \end{align*}
}

\subsubsection*{Vector normal principal $ \mathbf{N} $}

La dirección de $ \mathbf{N} $ de $ \frac{d\mathbf{T}}{ds} $. Observamos que
{
  \large
  \begin{align*}  
    \frac{d\mathbf{T}}{ds} &= \frac{1}{a^2 + b^2}(-a \cos t, -a \sin t, 0) \\
    \lvert\lvert \frac{d\mathbf{T}}{ds} \rvert\rvert &= \sqrt{
      \frac{(-a \cos t)^2 + (-a \sin t)^2 + 0^2}{(a^2+b^2)^2}
    } = \sqrt{
      \frac{a^2(\cos^2 t + \sin^2 t)}{(a^2 + b^2)^2}
    } \\ 
      \lvert\lvert \frac{d\mathbf{T}}{ds} \rvert\rvert &=
      \sqrt{
        \frac{a^2}{(a^2 + b^2)^2}
      } = \frac{a}{a^2 + b^2} \\ 
      \frac{
        \frac{d\mathbf{T}}{ds}
      }{
        \lvert\lvert \frac{d\mathbf{T}}{ds} \rvert\rvert
    } &= \frac{
        \frac{(-a \cos t -a \sin t + 0)}{a^2 + b^2}
      }{
        \frac{a}{a^2 + b^2} 
      } = \frac{(a^2 + b^2)(-a \cos t -a \sin t + 0)}{(a^2 + b^2)(a)} \\
      \frac{
        \frac{d\mathbf{T}}{ds}
      }{
        \lvert\lvert \frac{d\mathbf{T}}{ds} \rvert\rvert
      } &= \frac{a(- \cos t - \sin t + 0)}{a} = (- \cos t - \sin t + 0)
  \end{align*}
}
Entonces el vector unitario normal es:
{
  \large
  \begin{align*} 
    \mathbf{N}(t) =
    \frac{
        \frac{d\mathbf{T}}{ds}
      }{
        \lvert\lvert \frac{d\mathbf{T}}{ds} \rvert\rvert
      }
    = (- \cos t, - \sin t, 0)
  \end{align*}
}
(Nótese que $ \mathbf{N} $ ya es unitario porque $ \cos^2 t + \sin^2 t = 1 $)

\subsubsection*{Curvatura $ \kappa $}
Usamos la fórmula $ \kappa = \frac{\lvert r' \times r'' \rvert}{\lvert r' \rvert^3} $

Primero el producto vectorial:
{
  \large
  \begin{align*}
    r' \times r'' &= 
    \begin{vmatrix}
      i & j & k \\
      -a \sin t & a \cos t & b \\
      -a \cos t & -a \sin t & 0 \\
    \end{vmatrix} \\
    r' \times r'' &= i
    \begin{vmatrix}
      a \cos t & b  \\
      -a \sin t & 0 \\
    \end{vmatrix} - j
    \begin{vmatrix}
      -a \sin t & b \\
      -a \cos t & 0 \\
    \end{vmatrix} + k
    \begin{vmatrix}
      -a \sin t & a \cos t \\
      -a \cos t & -a \sin t \\
    \end{vmatrix} \\
    r' \times r'' &= i(ab \sin t) - j(ab \cos t) + k(a^2 \sin^2 t + a^2 \cos^2 t) \\
    r' \times r'' &= i(ab \sin t) - j(ab \cos t) + k(a^2 (\sin^2 t + \cos^2 t)) \\
    r' \times r'' &= i(ab \sin t) - j(ab \cos t) + k(a^2)
  \end{align*}
}
Su magnitud:
{
  \large
  \begin{align*}
    \lvert r' \times r'' \rvert &= \sqrt{(ab \sin t)^2 + (-ab \cos t)^2 + (a^2)^2} \\
    \lvert r' \times r'' \rvert &= \sqrt{(ab)^2 (\sin t + \cos t)^2 + (a^2)^2} =\sqrt{(ab)^2 (\sin^2 t + \cos^2 t) + a^4}
    \\
    \lvert r' \times r'' \rvert &= \sqrt{a^4 + (ab)^2} = \sqrt{a^2(a^2 + b^2)} = \sqrt{a^2}\sqrt{a^2+b^2} \\
    \lvert r' \times r'' \rvert &= a\sqrt{a^2+b^2}
  \end{align*}
}
y tambien $ \lvert r' \rvert $:
{
  \large
  \begin{align*}
    \lvert r' \rvert &= \sqrt{(-a \sin t)^2 + (a \cos t)^2 + b^2} = \sqrt{a^2(\sin^2 t + \cos^2 t) + b^2} \\
    \lvert r' \rvert &= \sqrt{a^2 + b^2}
  \end{align*}
}
Entonces:
{
  \large
  \begin{align*}
    \kappa &= \frac{a\sqrt{a^2 + b^2}}{(\sqrt{a^2 + b^2})^3} = \frac{a\sqrt{a^2 + b^2}}{
    (\sqrt{a^2 + b^2}) (\sqrt{a^2 + b^2})^2
  } = \frac{a}{(\sqrt{a^2 + b^2})^2} \\
  \kappa &= \frac{a}{a^2 + b^2}
  \end{align*}
}

\subsubsection*{Vector binormal $ \mathbf{B} $}
{
  \large
  \begin{align*}  
    \mathbf{B} = \mathbf{T} \times \mathbf{N}
  \end{align*}
}
Primero el producto vectorial:
{
  \large
  \begin{align*}  
    \lvert \mathbf{T} \times \mathbf{N} \rvert =
    \begin{vmatrix}
      i & j & k \\
      \frac{-a \sin t}{\sqrt{a^2 + b^2}} & \frac{a \cos t}{\sqrt{a^2 + b^2}} & \frac{b}{\sqrt{a^2 + b^2}} \\
      - \cos t & - \sin t & 0 \\
    \end{vmatrix}
  \end{align*}
  \begin{align*}
    \lvert \mathbf{T} \times \mathbf{N} \rvert &= 
    i
    \begin{vmatrix}
      \frac{-a \cos t}{\sqrt{a^2 + b^2}} & \frac{b}{\sqrt{a^2 + b^2}} \\
      - \sin t & 0 \\ 
    \end{vmatrix} 
    - j
    \begin{vmatrix}
      \frac{-a \sin t}{\sqrt{a^2 + b^2}} & \frac{b}{\sqrt{a^2 + b^2}} \\
      - \cos t & 0 \\ 
    \end{vmatrix}
    + k
    \begin{vmatrix}
      \frac{-a \sin t}{\sqrt{a^2 + b^2}} & \frac{a \cos t}{\sqrt{a^2 + b^2}} \\
      - \cos t & - \sin t \\ 
    \end{vmatrix} \\
    \lvert \mathbf{T} \times \mathbf{N} \rvert &=
    i
    (\frac{b \sin t}{\sqrt{a^2 + b^2}})
    - j
    (\frac{b \cos t}{\sqrt{a^2 + b^2}})
    + k
    (\frac{-a \sin^2 t}{\sqrt{a^2 + b^2}} + \frac{a \cos^2 t}{\sqrt{a^2 + b^2}}) \\
    \lvert \mathbf{T} \times \mathbf{N} \rvert &=
    i
    (\frac{b \sin t}{\sqrt{a^2 + b^2}})
    - j
    (\frac{b \cos t}{\sqrt{a^2 + b^2}})
    + k
    (\frac{a}{\sqrt{a^2 + b^2}}(\sin^2 + \cos^2)) \\
    \lvert \mathbf{T} \times \mathbf{N} \rvert &=
    i
    (\frac{b \sin t}{\sqrt{a^2 + b^2}})
    - j
    (\frac{b \cos t}{\sqrt{a^2 + b^2}})
    + k
    (\frac{a}{\sqrt{a^2 + b^2}}) \\
  \end{align*}
}
Entonces el vector binormal:
{
  \large
  \begin{align*}
    \mathbf{B}(t) = \frac{1}{\sqrt{a^2 + b^2}}(b \sin t, -b \cos t, a)
  \end{align*}
}
y verificaremos que sea unitario:
{
  \large
  \begin{align*}
    \lvert \mathbf{B}(t) \rvert &= \sqrt{
      \frac{
        (b \sin t)^2 + (-b \cos t)^2 + a^2
      }{
        (\sqrt{a^2 + b^2})^2
      }
    } = \sqrt{
      \frac{
        b^2 \sin^2 t + b^2 \cos^2 t + a^2
      }{
        \sqrt{a^2 + b^2}
      }
    } \\
    \lvert \mathbf{B}(t) \rvert &= \sqrt{
      \frac{
        b^2(\sin^2 t + \cos^2 t) + a^2
      }{
        a^2 + b^2
      }
    } = \sqrt{\frac{b^2 + a^2}{a^2 + b^2}} = \sqrt{1} = 1
  \end{align*}
}

\subsubsection*{Torsión $ \tau $}
Usamos la formula
{
  \large
  \begin{align*}
    \tau = \frac{(r' \times r'') \cdot r'''}{\lvert r' \times r'' \rvert^2}
  \end{align*}
}
Ya tenemos $ (r' \times r''') $ , $ r''' $ y $ \lvert r' \times r'' \rvert $:
{
  \large
  \begin{align*}
    r' \times r''' &= (ab \sin t, -ab \cos t, a^2) \\
    r''' &= (a \sin t, -a \cos t, 0) \\
    \lvert r' \times r'' \rvert &= a\sqrt{a^2 + b^2}
  \end{align*}
}
Calculamos el cuadrado de $ \lvert r' \times r'' \rvert $
{
  \large
  \begin{align*}
    \lvert r' \times r'' \rvert^2 &= (a\sqrt{a^2 + b^2})^2 = a^2(\sqrt{a^2 + b^2})^2 = a^2(a^2 + b^2)
  \end{align*}
}
Entonces
{
  \large
  \begin{align*}
    \tau &= \frac{(ab \sin t -ab \cos t + a^2)(a \sin t -a \cos t + 0)}{a^2(a^2 + b^2)} \\
    \tau &= \frac{ab \sin t \cdot a \sin t + (-ab \cos t)(-a \cos t) + a^2(0)}{a^2(a^2 + b^2)} \\
    \tau &= \frac{a^2b \sin^2 t + a^2b \cos^2 t}{a^2(a^2 + b^2)} = \frac{a^2b(\sin^2 t + \cos^2 t)}{a^2 (a^2 + b^2)}
    = \frac{a^2b}{a^2(a^2 + b^2)} \\
    \tau &=  \frac{b}{a^2 + b^2}
  \end{align*}
}

\subsubsection*{Verificaciones con las formulas de Frenet-Serret}
Las formulas de Frenet-Serret son:
{
  \large
  \begin{align*}
    \frac{d\mathbf{T}}{ds} &= \kappa \mathbf{N} \\ 
    \frac{d\mathbf{N}}{ds} &= -\kappa \mathbf{T} + \tau \mathbf{B} \\
    \frac{d\mathbf{B}}{ds} &= -\tau \mathbf{N}
  \end{align*}
}

para la primera formula ya hemos calculado $ \frac{d\mathbf{T}}{ds} $ :
{
  \large
  \begin{align*}
    \frac{d\mathbf{T}}{ds} = \kappa \mathbf{N}
  \end{align*}
  \begin{align*}  
    \frac{1}{a^2 + b^2}(-a \cos t, -a \sin t, 0) &= \frac{a}{a^2 + b^2} (- \cos t, - \sin t, 0) \\
    \frac{(-a \cos t, -a \sin t, 0)}{a^2 + b^2} &= \frac{(-a \cos t, -a \sin t, 0)}{a^2 + b^2}
  \end{align*}
}

para la segunda formula calcularemos $ \frac{d\mathbf{N}}{ds} $
{
  \large
  \begin{align*}
    \frac{d\mathbf{N}}{ds} &= -\kappa \mathbf{T} + \tau \mathbf{B} \\
  \end{align*}
  \begin{align*}
    \frac{d\mathbf{N}}{ds} &= \frac{1}{\lvert r' \rvert} \frac{d\mathbf{N}}{dt} = 
    \frac{1}{\sqrt{a^2 + b^2}}(\sin t, -\cos t, 0)
  \end{align*}
}

calculo de $ -\kappa\,\mathbf{T} + \tau\,\mathbf{B} $
{
  \begin{align*}
    -\kappa\,\mathbf{T} + \tau\,\mathbf{B}
    &= -\frac{a}{a^2+b^2}\frac{1}{\sqrt{a^2+b^2}}(-a\sin t,\, a\cos t,\, b)
    + \frac{b}{a^2+b^2}\frac{1}{\sqrt{a^2+b^2}}(b\sin t,\, -b\cos t,\, a)
    \\
    -\kappa\,\mathbf{T} + \tau\,\mathbf{B}
    &= \frac{1}{(a^2+b^2)^{3/2}}
    \big(
    a^2\sin t + b^2\sin t,\;
    - a^2\cos t - b^2\cos t,\;
    -ab + ab
    \big) \\
    -\kappa\,\mathbf{T} + \tau\,\mathbf{B} &= \frac{1}{\sqrt{a^2+b^2}}(\sin t,\; -\cos t,\; 0).
\end{align*}
}

Entonces:
{
  \large
  \begin{align*}
    \frac{d\mathbf{N}}{ds} &= -\kappa \mathbf{T} + \tau \mathbf{B} \\ 
    \frac{1}{\sqrt{a^2+b^2}}(\sin t,\; -\cos t,\; 0) &= \frac{1}{\sqrt{a^2+b^2}}(\sin t,\; -\cos t,\; 0)
  \end{align*}
}

Para la tercera formula calcularemos $ \frac{d\mathbf{B}}{ds} $
{
  \large
  \begin{align*}
    \frac{d\mathbf{B}}{ds} &= -\tau \mathbf{N}
  \end{align*}
  \begin{align*}
    \frac{d\mathbf{B}}{ds} = \frac{1}{\lvert r' \rvert} \frac{d\mathbf{B}}{dt} = 
    \frac{1}{a^2 + b^2}(b \cos t, b \sin t, 0)
  \end{align*}
}
calculo de $ -\tau \mathbf{N}  $

{
  \large
  \begin{align*}
    -\tau \mathbf{N} &= -\frac{b}{a^2 + b^2}(- \cos t, - \sin t, 0) \\
    -\tau \mathbf{N} &= \frac{1}{a^2 + b^2}(b \cos t, b \sin t, 0)
  \end{align*}
}

Entonces
{
  \large
  \begin{align*}
    \frac{d\mathbf{B}}{ds} &= -\tau \mathbf{N} \\
    \frac{1}{a^2 + b^2}(b \cos t, b \sin t, 0) &= \frac{1}{a^2 + b^2}(b \cos t, b \sin t, 0)
  \end{align*}
}

\subsubsection*{Conclusion}
Para la hélice $ r(t) = (a \cos t, a \sin t, bt) $
{
  \large
  \begin{itemize}
    \item $ \lvert r' \rvert = \sqrt{a^2 + b^2} $
    \item $ \mathbf{T}(t) = \frac{1}{\sqrt{a^2 + b^2}}(-a \sin t, a \cos t, b) $
    \item $ \mathbf{N}(t) = (- \cos t, - \sin t, 0) $
    \item $ \mathbf{B}(t) = \frac{1}{\sqrt{a^2 + b^2}}(b \sin t, -b \cos t, a) $
    \item $ \kappa = \frac{a}{a^2 + b^2}$
    \item $ \tau = \frac{b}{a^2 + b^2} $
  \end{itemize}
}

Con esto concluimos que las 3 formulas de Frenet-Serret se cumplen y obtenemos el Tiedro de Frenet:
{
  \begin{figure}  
    \begin{center}
    \includegraphics[width=15cm, height=15cm]{geogebra-export}
    \includegraphics[width=10cm, height=10cm]{geogebra}
    \end{center}
  \end{figure}
}
\newpage

\subsection*{Aplicaciones}

\end{document}
