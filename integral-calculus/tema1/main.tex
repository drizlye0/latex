\documentclass[12pt, a4paper, oneside]{article}
\usepackage{amsmath, amssymb, amsthm}
\setlength{\parindent}{0pt}

\begin{document}
\subsection*{Problema: El área bajo una parábola truncada}

\textbf{Resuelve el siguiente problema:}

Una empresa de ingeniería está analizando el flujo de agua que pasa por una compuerta cuya forma puede modelarse con la función

\begingroup
\Large
\begin{equation*}
	f(x) = 4 - x^2
\end{equation*}
\endgroup

en el intervalo $[-1, 2]$, donde $x$ representa metros a lo largo de la base horizontal, y $f(x)$ la altura del flujo de agua (en metros).

Tu tarea es:

\begin{enumerate}
	\item Dibuja la grafica (Puedes utilizar una graficadora) y encuentra el área utilizando el concepto de figuras amorfas con base en los rectángulos de 0.5cm

	\item Estimar el área bajo la curva $f(x) = 4 - x^2$  en el intervalo $[-1, 2]$ usando una suma de Riemann con $n = 6$ subintervalos. Usa el punto  medio de cada subintervalo para la altura de los rectángulos.

	\item Calcular el área exacta bajo la curva en el mismo intervalo usando la integral definida.

	\item Comparar los resultados y explicar por qué hay diferencia (o no).
\end{enumerate}

\end{document}
